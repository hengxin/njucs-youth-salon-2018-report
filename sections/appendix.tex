% file: sections/appendix.tex

\section{附录}

\appendix

% file: sections/bib.tex

%%%%%%%%%%%%%%%%%%%%
\begin{frame}[allowframebreaks]
  \printbibliography
\end{frame}
%%%%%%%%%%%%%%%%%%%%


%%%%%%%%%%%%%%%%%%%% Begin: Jupiter %%%%%%%%%%%%%%%%%%%%

%%%%%%%%%%%%%%%%%%%%
\begin{frame}{}
  \centerline{\large 针对列表的操作转换函数~\ncite{Ellis:SIGMOD89}}

  \resizebox{\textwidth}{!}{
    \begin{minipage}{\textwidth}
      \input{sections/list-ot}
    \end{minipage}
  }
\end{frame}
%%%%%%%%%%%%%%%%%%%%
%%%%%%%%%%%%%%%%%%%% End: Jupiter %%%%%%%%%%%%%%%%%%%%

%%%%%%%%%%%%%%%%%%%% Begin: VPC %%%%%%%%%%%%%%%%%%%%

%%%%%%%%%%%%%%%
\begin{frame}{}
  \begin{ctheorem}[\rwclosure{} 算法正确性]
    \begin{center}
      \emph{\vpc{mu}} 实例满足 \emph{\PRAM{}} 一致性 \\[5pt]
      $\iff$ \\[5pt]
      \rwclosure{} 算法所得图是 \emph{DAG} 图
    \end{center}
  \end{ctheorem}

  \pause
  \vspace{0.20cm}

  \begin{cproof}
    \begin{description}
      \item[``$\Longrightarrow$''] 反证法
      \item[``$\Longleftarrow$''] 
	\textcolor{red}{难点:} DAG 图蕴含着多个全序

	\textcolor{red}{技巧:} 对读操作作数学归纳, 构造合法调度
    \end{description}
  \end{cproof}

  \pause
  \vspace{0.30cm}

  \rwclosure{} 算法复杂度: 
  \[
    \underbrace{O(n^2)}_{\textrm{\#loops}} \quad\cdot
	\underbrace{O(n^3)}_{\textrm{transitive closure}}  = O(n^5)
  \]
\end{frame}
%%%%%%%%%%%%%%%

%%%%%%%%%%%%%%%
\begin{frame}{}
  \readcentric{} 算法复杂度: 
  \[
    \underbrace{O(n)}_{\textrm{iterations}} \cdot
	\underbrace{O(n \cdot n^2)}_{\textcolor{red}{\textsc{Topo-Schedule}}} = O(n^4)
  \]

  \vspace{0.30cm}
  \begin{clemma}[\textsc{Topo-Schedule} 的非迭代性]
    \begin{center}
      设\textsc{Topo-Schedule} 正在处理读操作$r$,\\
      则\textcolor{blue}{局部子图}中的每个写操作\textcolor{red}{最多只有一次机会}\\
      在满足规则 $w'wr$ 的三元组中扮演``$w'$角色''。
    \end{center}
  \end{clemma}
\end{frame}
%%%%%%%%%%%%%%%

%%%%%%%%%%%%%%%
\begin{frame}{实验评估}
  实验目的~\footnotemark[1]~\footnotetext[1]{机器配置: Intel Core i7 3.40GHZ, 4GB RAM.}:
  \begin{enumerate}
    \item 考察 \readcentric{} 算法的实际效率 
      \textcolor{blue}{\small ({\it vs.} 渐近时间复杂度)}
    \item 对比 \readcentric{} 算法与 \rwclosure{} 算法的效率
  \end{enumerate}

  \pause
  \vspace{0.50cm}

  两类负载:
  \begin{enumerate}
    \item 随机生成的系统执行
    \item 满足 \PRAM{} 一致性的系统执行 \textcolor{red}{\small ($\approx$ 最坏情况输入)}
  \end{enumerate}
\end{frame}
%%%%%%%%%%%%%%%

%%%%%%%%%%%%%%%
\begin{frame}{}
  \begin{figure}[t]
    \centering
    \begin{subfigure}[t]{0.50\textwidth}
      \includegraphics[width = 0.80\textwidth]{figs/vpc-random-cmp.pdf}
    \end{subfigure}%
    ~
    \begin{subfigure}[t]{0.50\textwidth}
      \includegraphics[width = 0.80\textwidth]{figs/vpc-valid-cmp.pdf}
    \end{subfigure}
    \caption{\rwclosure{} 算法与 \readcentric{} 算法在
    \textcolor{blue!80}{ (左) 随机生成}的执行及
    \textcolor{red!80}{ (右) 满足 \PRAM{} 一致性}的执行上的运行时间。}
  \end{figure}

  \pause
  \begin{center}
    \textcolor{red}{(右)} 20个进程、8,000 个操作: 

    \readcentric{} 可获得 694 倍加速.
  \end{center}
\end{frame}
%%%%%%%%%%%%%%%

%%%%%%%%%%%%%%%
\begin{frame}{}
  \fig{width = 0.45\textwidth}{figs/vpc-scalability-more.pdf}
  {\readcentric{} 算法在满足 \PRAM{} 一致性的执行上的运行时间}

  \vspace{-0.30cm}

  \begin{description}
    \centering
    \item[\readcentric{}:] 20个进程、60,000个操作 < 600s~\footnotemark[1]~\footnotetext[1]{用于测试, 规模可用}
    \item[\rwclosure{}:] 20个进程、8,000个操作 > 3,000s
  \end{description}
\end{frame}
%%%%%%%%%%%%%%%

%%%%%%%%%%%%%%%
\begin{frame}{}
  \begin{center}
    \hl{\large $\mathcal{I}$-Atomicity = \red{$i$-实时序} + 读写语义}
  \end{center}

  \vspace{0.60cm}
  \centerline{$\mathcal{I}$: Inversions}

  \pause
  \vspace{0.30cm}
  \[
    f\Big(\set{\text{inversions}}\Big) \le i
  \]

  \vspace{0.50cm}
  \centerline{\large 定义\red{框架}}
\end{frame}
%%%%%%%%%%%%%%%

%%%%%%%%%%%%%%%
\begin{frame}{}
  \begin{center}
    \resizebox{0.50\textwidth}{!}{% file: verification-quantification-ruler.tex

\def\len{4}

\begin{tikzpicture}
  % ruler
  \draw[thick, {|[right, scale = 2.5]}-{|[left, scale = 2.5]}, blue] (0,0) node (l) [below = 8pt] {$0$} 
  	to (\len,0) node (r) [below = 8pt] {$1$};

  \foreach \pos in {1,2,...,9} {
    \draw[red, thick] ($(0.1 * \len * \pos, 0)$) node (\pos) {} to ++(0, 5pt);
  }

  % verification
  \node (veri) [below = 1.00cm of 5, blue] {Verification};
  \path[blue] (l) edge (veri.west) 
	      (r) edge (veri.east);

  % quantification
  \uncover<2->{
    \node (quan) [red] at ($(l)!0.5!(r)$) {Quantification};
    \path[red] (quan) edge[->|] (l.center)
		      edge[->|] (r.center);
  }
\end{tikzpicture}
}
  \end{center}

  \vspace{0.10cm}
  \uncover<2->{
    \begin{center}
      \red{\large 协议量化分析} \\[5pt]
      \ncite{Lee:DC05} \ncite{Bailis:VLDB12} \ncite{Chatterjee:PODC17}
    \end{center}
  }

  \vspace{0.50cm}
  \uncover<3->{
    \centerline{\hl{\large PA2AM: Probabilistically-Atomic $2$-Atomicity}}
  }
\end{frame}
%%%%%%%%%%%%%%%
%%%%%%%%%%%%%%%%%%%% End: VPC %%%%%%%%%%%%%%%%%%%%

%%%%%%%%%%%%%%%%%%%% Begin: Future Work %%%%%%%%%%%%%%%%%%%%
%%%%%%%%%%%%%%%
\begin{frame}{}
  \begin{center}
    ($50$种) \hl{一致性模型}关系图 \ncite{Viotti:CSUR16} \ncite{Burckhardt:Book14}
  \end{center}

  \fignocaption{width = 0.95\textwidth}{figs/non-transactional-consistency-models}

  \centerline{\red{\large 建立统一的形式化框架}}
\end{frame}
%%%%%%%%%%%%%%%%%%%% End: Future Work %%%%%%%%%%%%%%%%%%%%
