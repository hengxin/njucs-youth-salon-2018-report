% file: sections/intro.tex

\section{研究背景}

%%%%%%%%%%%%%%%
\begin{frame}{}
  \begin{center}
    {\large Abstract Data Types} \ncite{Liskov:VHLL74} \\[8pt]

    (单线程; 顺序语义)
  \end{center}

  \begin{columns}
    \column{0.40\textwidth}
      \fignocaption{width = 0.60\textwidth}{figs/alg-ds-wirth}
    \pause
    \column{0.40\textwidth}
      \fignocaption{width = 0.25\textwidth}{figs/adt}
  \end{columns}
\end{frame}
%%%%%%%%%%%%%%%

%%%%%%%%%%%%%%%
\begin{frame}{}
  \begin{center}
    {\large Concurrent Data Types} \ncite{Herlihy:TOPLAS90} \\[8pt]

    (多线程; 并发语义)
  \end{center}

  \begin{columns}
    \column{0.40\textwidth}
      \fignocaption{width = 0.70\textwidth}{figs/taomp-herlihy}
    \pause
    \column{0.50\textwidth}
      \fignocaption{width = 0.45\textwidth}{figs/cdt}

      \only<3->{
	\begin{center}
	  \hl{\Large PL} {(Programming Language)}
	\end{center}
      }
  \end{columns}
\end{frame}
%%%%%%%%%%%%%%%

%%%%%%%%%%%%%%%
\begin{frame}{}
  \begin{center}
    \hl{\large Replicated Data Types} \ncite{Shapiro:TR11} \ncite{Burckhardt:POPL14} \\[8pt]

    (多副本; 复制语义)
  \end{center}

  \pause
  \fignocaption{width = 0.30\textwidth}{figs/rdt}

  \pause
  \begin{center}
    \hl{\Large DC} {(Distributed Computing)}
  \end{center}
\end{frame}
%%%%%%%%%%%%%%%

%%%%%%%%%%%%%%%
% \begin{frame}{}
%   \begin{columns}
%     \column{0.50\textwidth}
%       \fignocaption{width = 0.60\textwidth}{figs/keep-calm-why-bother}
%     \pause
%     \column{0.50\textwidth}
%       \centerline{\hl{\red{\Huge 新平台}}}
%   \end{columns}
% \end{frame}
%%%%%%%%%%%%%%%

%%%%%%%%%%%%%%%
\begin{frame}{}
  \begin{center}
    {\hl{\Large 新平台: 大规模分布式系统}}
  \end{center}

  \fignocaption{width = 0.50\textwidth}{figs/sina-weibo-world-map.pdf}

  \pause
  \vspace{0.10cm}
  \begin{center}
    低延迟 \quad 高可用性 {\small (4个9)} \quad 高容错性 \quad 高可扩展性
  \end{center}

  % \begin{columns}
  %   \column{0.40\textwidth}
  %   新浪微博社交应用~\footnotemark:
  %   \begin{itemize}
  %     \item 日均用户近一亿名
  %     \item 日均消息近一亿条
  %   \end{itemize}
  %   \pause
  %   \column{0.50\textwidth}
  %   特性需求: 
  %   \begin{itemize}
  %     \item 低延迟, 高可用性 (4个9~\footnotemark)
  %     \item 高容错性, 高可扩展性
  %   \end{itemize}
  % \end{columns}
  % 
  % \footnotetext[1]{2015第三季度; 数据来自 \href{https://www.chinainternetwatch.com/15740/weibo-q3-2015/}{China Internet Watch}.}
  % % See http://tex.stackexchange.com/a/340079/23098
  % \alt<1>{\let\thefootnote\relax\footnotetext{~}}{\footnotetext[2]{数据来自 \href{http://www.infoq.com/cn/articles/weibo-platform-availability-9999}{InfoQ}.}}
\end{frame}
%%%%%%%%%%%%%%%
\begin{frame}{}
  \graphicspath{{tikz/}}
  \begin{figure}[h!]
    \centering
    \begin{adjustbox}{max totalsize = {0.50\textwidth}{1.00\textheight}, center}
      \input{tikz/distributed-data-overlay-for-background}
    \end{adjustbox}
  \end{figure}

  \vspace{0.20cm}
  \begin{center}
    \begin{minipage}{0.65\textwidth}
      \red{\large 分布数据 \term{distributed data}:}

      \vspace{0.20cm}
      \begin{enumerate}
	\item<2-> 分区 \term{partition}: 水平扩展
	\item<3-> \hl{副本 \term{replication}}: 就近访问, 容灾备份
      \end{enumerate}
    \end{minipage}
  \end{center}
\end{frame}
%%%%%%%%%%%%%%%

%%%%%%%%%%%%%%%
\begin{frame}{}
  \begin{center}
    \begin{minipage}{0.50\textwidth}
      {\large 复制数据类型} \ncite{Shapiro:TR11} 

      \vspace{0.20cm}
      \begin{itemize}
	\setlength{\itemsep}{4pt}
	\item Read/Write Register
	\item Counter
	\item Set
	\item List
	\item HashMap
	\item Disjoint Set
	\item Graph
	\item $\dots$
      \end{itemize}
    \end{minipage}
  \end{center}
\end{frame}
%%%%%%%%%%%%%%%

%%%%%%%%%%%%%%%
\begin{frame}{}
  \begin{columns}
    \column{0.40\textwidth}
      \fignocaption{width = 0.98\textwidth}{figs/what-is-new}
    \column{0.60\textwidth}
      \centerline{\hl{\red{\Huge 新问题, 新挑战}}}
  \end{columns}

  \pause
  \vspace{0.30cm}
  \fignocaption{frame, width = 0.80\textwidth}{figs/rdt-popl}
  \centerline{\ncite{Burckhardt:POPL14}}
\end{frame}
%%%%%%%%%%%%%%%

%%%%%%%%%%%%%%%
\begin{frame}{}
  \begin{center}
    \resizebox{0.65\textwidth}{!}{\input{tikz/rdt-research-two-work}}
  \end{center}
\end{frame}
%%%%%%%%%%%%%%%
