% file: sections/vpc-work.tex

%%%%%%%%%%%%%%%
\begin{frame}{}
  \begin{center}
    \vspace{0.50cm}
    协议验证 (Verification of a Protocol) \\[5pt]
    \ncite{Bouajjani:POPL14} \ncite{Bouajjani:POPL17}

    \vspace{0.30cm}
    \hl{执行验证 (Verification of an Execution)}

    \pause
    \vspace{0.50cm}

    \fignocaption{width = 0.50\textwidth}{figs/sla}
    
    \vspace{0.30cm}
    {黑盒测试/确认系统是否提供了其所声称的数据一致性}

    \ncite{Amazon:SOSP07} \ncite{Golab:PODC11}
  \end{center}
\end{frame}
%%%%%%%%%%%%%%%

%%%%%%%%%%%%%%%
\begin{frame}{}
  \begin{description}
    \item[\red{\large PRAM:}] 包含存储系统常提供的最基本的``会话'' \term{session} 一致性

	\ncite{Terry:PDIS94} \ncite{Brzezinski:PDP04}
  \end{description}

  \fignocaption{width = 0.60\textwidth}{figs/pram-monotonic-writes-example}
  \vspace{-0.10cm}
  {\centerline{\PRAM{} 保证``单调写''性质}}
\end{frame}
%%%%%%%%%%%%%%%

%%%%%%%%%%%%%%%
\begin{frame}{}
  \begin{cdef}[VPC (Verifying PRAM Consistency) 判定问题]
    \vspace{8pt}
    \begin{description}
      \setlength{\itemsep}{8pt}
      \item[实例:] 系统执行 \emph{\small (execution $e$)}
      \item[问题:] 该执行 $e$ 是否满足 \emph{PRAM} 一致性模型 \emph{\small ($\mathcal{C}$)}? 
    \end{description}    

    \[
      e \in \mathcal{C} \Rightarrow \set{0,1}?
    \]
  \end{cdef}
\end{frame}
%%%%%%%%%%%%%%%

%%%%%%%%%%%%%%%
\begin{frame}{}
  \begin{cdef}[系统执行]
    系统执行 $e \triangleq \{h_p \mid h_p: \text{进程 } \;p \text{ 上的读写操作序列}\}$

    \vspace{0.30cm}
    \textcolor{blue}{规模 $n$:} 系统执行中读写操作的总数
  \end{cdef}

  \vspace{0.50cm}
  \fignocaption{width = 0.70\textwidth}{figs/execution-example.pdf}
\end{frame}
%%%%%%%%%%%%%%%

%%%%%%%%%%%%%%%
\begin{frame}{}
  \begin{cdef}[\PRAM{} 一致性模型]
    \begin{center}
      系统执行 $e$ 满足 \emph{\PRAM{}} 一致性 \\[5pt]
      $\iff$ \\[5pt]
      $\forall p:$ $p$ 上\textcolor{blue}{所有操作}与其它进程上所有\textcolor{blue}{写操作}存在\textcolor{red}{合法}调度
    \end{center}
  \end{cdef}

  \vspace{0.30cm}
  % \fignocaption{width = 0.65\textwidth}{figs/execution-example.pdf}
  \begin{center}
    \resizebox{0.65\textwidth}{!}{\input{tikz/execution-example}}
  \end{center}

  \pause
  \vspace{-0.80cm}

  \begin{gather*}
    p_{0}: Wf2\; Wf1\; Wz2\; \textcolor<3>{blue}{Wz1}\; Wy2\; Wy1\; \textcolor{brown}{Rf1} 
    Wx5\; Wx3\; \textcolor<4>{red}{Wx2}\; Wc1\; \textcolor{brown}{Rc1} \\
    \textcolor{brown}{Rz1} \textcolor{brown}{Ry1}
    Wa1\; \textcolor{brown}{Ra1} \textcolor<3>{blue}{Wb1}\; \textcolor{brown}{Rb1} \textcolor<4>{red}{\textcolor<1-3>{brown}{Rx2}}
  \end{gather*}
\end{frame}
%%%%%%%%%%%%%%%

%%%%%%%%%%%%%%%
\begin{frame}{}
  \begin{table}[!t]
    \centering
    \caption{VPC 问题的四种变体 (按``执行''的类型) 及复杂度 \\ \red{\small ($[\ast]:$ 本文工作)}}
    \renewcommand\arraystretch{1.2}
    \resizebox{\textwidth}{!}{%
      \begin{tabular}{|c|c|c|}
	\hline
	& \it (S)ingle variable  & \it (M)ultiple variables
	\\ \hline
	    {\it write (D)uplicate values} &
	    \innercell{c}{VPC-SD \\ \uncover<2->{\textcolor{red}{\small (\npc{}) $[\ast]$}}} &
	    \innercell{c}{VPC-MD \\ \uncover<2->{\textcolor{red}{\small (\npc{}) $[\ast]$}}}
	\\ \hline
	    \only<1-2>{\it write (U)nique value}\only<3>{\cellcolor{brown!80}{\it write (U)nique value}} &
	    \innercell{c}{VPC-SU \\ \uncover<2->{\textcolor{blue}{\small (P)}} \\ \uncover<2->{\ncite{Golab:PODC11}}} &
	    \innercell{c}{VPC-MU \\ \uncover<2->{\textcolor{red}{\small (P) $[\ast]$}}}
	\\ \hline
      \end{tabular}
    }
  \end{table}

  \vspace{10pt}
  \uncover<3->{\textcolor{brown}{\centerline{Read-mapping \ncite{Gibbons:SICOMP97}: $\forall r,\; \exists! w,\; f(r) = w$.}}}
\end{frame}
%%%%%%%%%%%%%%%

%%%%%%%%%%%%%%%
\begin{frame}{}
  \begin{center}
    \hl{\large VPC-SD (VPC-MD) 是 \npc{} 问题}
  \end{center}
  % 多项式归约: 从 \textsc{Unary 3-Partition} 到 \vpc{sd}

  \pause
  \fig{width = 0.40\textwidth}{figs/vpcsd-npc}
  {\textsc{Unary 3-Partition} 实例 $A = \{2,2,1,1,1,1\}, m = 2, B = 4$ 对应的 VPC-SD 执行} 
\end{frame}
%%%%%%%%%%%%%%%

%%%%%%%%%%%%%%%
\begin{frame}{}
  \begin{center}
    \fignocaption{width = 0.40\textwidth}{figs/mdb-stack}

    \pause
    \vspace{0.40cm}
    \begin{quote}
      ``Basically I'm a programmer :-)  \\[8pt] \pause

      I enjoy reading papers about the complexity of puzzle games, \\[2pt]

      and I'm writing (amateur) proofs on the complexity of a few puzzle games.''
    \end{quote}

    \pause
    \vspace{0.30cm}
    \fignocaption{width = 0.90\textwidth}{figs/nearly42}
  \end{center}
\end{frame}
%%%%%%%%%%%%%%%

%%%%%%%%%%%%%%%
\begin{frame}{}
  \begin{center}
    {\hl{\large VPC-MU 的多项式算法 \rwclosure{}}}
  \end{center}

  \begin{figure}[h!]
    \centering
    \begin{adjustbox}{max totalsize = {0.60\textwidth}{1.00\textheight}, center}
      \input{tikz/rw-closure-example}
    \end{adjustbox}
    \caption{\rwclosure{} 算法示例: 在传递闭包之上迭代应用 $w'wr$ 规则}
  \end{figure}

  \vspace{-0.30cm}
  \uncover<4->{\fignocaption{width = 0.25\textwidth}{figs/wprimewr-order.pdf}}
\end{frame}
%%%%%%%%%%%%%%%

%%%%%%%%%%%%%%%
\begin{frame}{}
  \begin{ctheorem}[\rwclosure{} 算法正确性]
    \begin{center}
      \emph{\vpc{mu}} 实例满足 \emph{\PRAM{}} 一致性 \\[5pt]
      $\iff$ \\[5pt]
      \rwclosure{} 算法所得图是 \emph{DAG} 图
    \end{center}
  \end{ctheorem}

  \pause
  \vspace{0.20cm}

  \begin{cproof}
    \begin{description}
      \item[``$\Longrightarrow$''] 反证法
      \item[``$\Longleftarrow$''] \hl{对读操作作数学归纳}, 构造合法调度
    \end{description}
  \end{cproof}

  \pause
  \vspace{0.50cm}
  \rwclosure{} 算法复杂度: 
  \[
    \underbrace{O(n^2)}_{\textrm{\#loops}} \quad\cdot
	\underbrace{O(n^3)}_{\textrm{transitive closure}}  = O(n^5)
  \]
\end{frame}
%%%%%%%%%%%%%%%

%%%%%%%%%%%%%%%
\begin{frame}{}
  \rwclosure{} 算法的缺点:
  \begin{itemize}
	\item 在全图上应用 $w'wr$ 规则
	\item 应用 $w'wr$ 规则无特定顺序
  \end{itemize}

  \pause
  \vspace{0.80cm}

  \hl{VPC-MU 的多项式算法 \readcentric{} 要点:}
  \begin{itemize}
    \item \textcolor{blue}{增量式}调度每个读操作
    \item 在读操作诱导的\textcolor{blue}{局部子图}上按\textcolor{blue}{逆拓扑序}应用 $w'wr$ 规则
  \end{itemize}
\end{frame}
%%%%%%%%%%%%%%%

%%%%%%%%%%%%%%%
\begin{frame}{}
  \begin{ctheorem}[\readcentric{} 算法正确性]
    \begin{center}
      \emph{\vpc{mu}} 实例满足 \emph{\PRAM{}} 一致性 \\[5pt]
      $\iff$ \\[5pt]
      \readcentric{} 算法所得图是 \emph{DAG} 图
    \end{center}
  \end{ctheorem}

  \pause
  \vspace{0.30cm}
  \begin{cproof}
    \[
      \readcentric{}\;\; \purple{\xLeftrightarrow{\quad Reachability \quad}} \;\;\rwclosure{}
    \]
  \end{cproof}

  \pause
  \vspace{0.60cm}
  \readcentric{} 算法复杂度: 
  \[
    \underbrace{O(n)}_{\textrm{\blue{\#reads}}} \cdot
    \underbrace{O(n \cdot n^2)}_{\red{\textsc{Topo-Schedule}}} = O(n^4)
  \]
\end{frame}
%%%%%%%%%%%%%%%
