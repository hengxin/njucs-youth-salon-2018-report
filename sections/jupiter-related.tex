% file: sections/jupiter-related.tex

%%%%%%%%%%%%%%%%%%%%
\begin{frame}{}
  \centerline{\large \hl{个人体会:} 基于 OT 思想的协议晦涩难懂}

  \vspace{0.60cm}
  \begin{columns}
    \column{0.50\textwidth}
      \fignocaption{width = 0.80\textwidth}{figs/thread}
    \column{0.50\textwidth}
      \pause
      \begin{itemize}
	\setlength{\itemsep}{10pt}
	\centering
	\item 协议多种多样
	\item 经常不加证明
	\item 证明是错误的
	\item<3-> \hl{\red{勘误也是错的}}
      \end{itemize}
  \end{columns}
\end{frame}
%%%%%%%%%%%%%%%%%%%%

%%%%%%%%%%%%%%%%%%%%
\begin{frame}{}
  \begin{center}
    {\large 
      Model Checking/Verifying: 使用 TLA+/TLAPS \\[5pt]
      \href{https://github.com/Disalg-ICS-NJU/tlaplus-lamport-projects/tree/master/tlaplus-projects/Hengfeng-Wei/Wei-jupiter-tla}{\purple{\underline{jupiter-tlaplus@github}}}
    }
  \end{center}

  \begin{columns}
    \column{0.30\textwidth}
      \fignocaption{width = 0.80\textwidth}{figs/tlaplus}
      \vspace{-0.50cm}
      \fignocaption{width = 0.60\textwidth}{figs/lamport}
    \pause
    \column{0.50\textwidth}
      \fignocaption{width = 1.00\textwidth}{figs/jupiter-tlaplus}
  \end{columns}
\end{frame}
%%%%%%%%%%%%%%%%%%%%

