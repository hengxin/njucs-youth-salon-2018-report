% file: sections/jupiter-work.tex

%%%%%%%%%%%%%%%%%%%%
\begin{frame}{}
  \begin{center}
    \begin{mdframed}[frametitle = {\large 复制列表数据类型~\footnotemark}, frametitlerule = true, frametitlebackgroundcolor = brown!20,
      frametitleaboveskip = 8pt, frametitlebelowskip = 8pt, innertopmargin = 10pt]
      {\large 实现复制列表的 \blue{Jupiter 协议}~\ncite{Nichols:UIST95}\footfullcite{Nichols:UIST95} \red{满足} \\
      \blue{weak list specification}~\ncite{Attiya:PODC16}\footfullcite{Attiya:PODC16}.} \\[15pt]
    \end{mdframed}

    \footnotetext{\normalsize \uncover<2->{藏在脚注里的}\red{猜想}@PODC'2016~\ncite{Attiya:PODC16}}
  \end{center}
\end{frame}
%%%%%%%%%%%%%%%%%%%%

%%%%%%%%%%%%%%%%%%%%
\begin{frame}{}
  \fignocaption{width = 0.30\textwidth}{figs/outline}
  \begin{center}
    \hl{\large 提纲}
  \end{center}

  \pause
  \begin{columns}
    \column{0.60\textwidth}
      \begin{enumerate}[<+->]
	\setlength{\itemsep}{12pt}
	\item 为什么要研究复制列表数据类型?
	\item 什么是 \red{weak list specification} \wlspec{}?
	\item \red{Jupiter} 协议是如何工作的?
	\item 如何证明Jupiter满足\wlspec{}?
      \end{enumerate}
  \end{columns}
\end{frame}
%%%%%%%%%%%%%%%%%%%%

%%%%%%%%%%%%%%%%%%%%
\begin{frame}{}
  \centerline{\teal{\Large 基于\red{副本}的协同文本编辑系统}}

  % \vspace{-0.20cm}
  \fignocaption{width = 0.55\textwidth}{figs/coeditor}

  \pause
  \vspace{-0.40cm}
  \begin{center}
    \begin{itemize}
      \centering
      \item 副本节点要\red{立即}响应本地用户操作 \\[4pt]
      \item 更新操作\red{异步}传播到其它副本节点
    \end{itemize}
  \end{center}
\end{frame}
%%%%%%%%%%%%%%%%%%%%

%%%%%%%%%%%%%%%%%%%%
\begin{frame}{}
  \centerline{\Large \red{复制列表对象}: 建模编辑系统的核心功能}
  \vspace{0.30cm}

  \begin{center}
    \begin{minipage}{0.70\textwidth}
      \begin{description}
	\setlength{\itemsep}{10pt}
	\item[$\textsc{Ins}(a, p):$] 在 $p$ 位置插入元素 $a$
	\item[$\textsc{Del}(p):$] 删除 $p$ 位置上的元素
	\item[$\textsc{Read}:$] 返回该列表
      \end{description}
    \end{minipage}
  \end{center}
\end{frame}
%%%%%%%%%%%%%%%%%%%%

%%%%%%%%%%%%%%%%%%%%
\begin{frame}{}
  \begin{cdef}[最终收敛性 (Eventual Convergence)~\ncite{Ellis:SIGMOD89}]
    当系统处于``静默'' \teal{\emph{(Quiescence)}} 状态时, 所有副本节点上的列表是相同的。
  \end{cdef}

  \vspace{0.30cm}
  \begin{cdef}[强最终一致性 (Strong Eventual Consistency)~\ncite{Shapiro:SSS11}]
    如果两个副本节点处理了同一组更新操作, 则它们的列表是相同的。
  \end{cdef}

  \pause
  \vspace{0.60cm}
  \centerline{\red{\large 对系统的\red{中间状态}缺少足够的约束}}
\end{frame}
%%%%%%%%%%%%%%%%%%%%

%%%%%%%%%%%%%%%%%%%%
\begin{frame}{}
  \fignocaption{width = 0.80\textwidth, frame}{figs/podc16-attiya}

  \vspace{0.20cm}
  \begin{cdef}[Weak List Specification \wlspec{}~\ncite{Attiya:PODC16}]
    Informally, \wlspec{} requires the ordering between \red{elements that are not deleted} to be consistent across the system.
  \end{cdef}

  \vspace{0.60cm}
  \centerline{\teal{\large 定义在系统所有列表状态上的\red{全局}性质}}
\end{frame}
%%%%%%%%%%%%%%%%%%%%

%%%%%%%%%%%%%%%%%%%%
\begin{frame}{}
  \begin{center}
    {\large \wlspec{} 等价于:}
  \end{center}

  \begin{cdef}[\hl{状态对兼容性} (Pairwise State Compatibility Property)]
    任给两个列表状态 $\sigma_0$、$\sigma_1$, 若它们含有两个共同元素 $a$、$b$, \\
    则 $a$、$b$ 在 $\sigma_0$ 与 $\sigma_1$ 中的相对顺序保持一致。
  \end{cdef}

  \vspace{0.30cm}
  \begin{columns}
    \column{0.50\textwidth}
      \fignocaption{width = 0.50\textwidth}{figs/ex-weak-list-spec}
      \vspace{-0.60cm}
      \fignocaption{width = 0.30\textwidth}{figs/red-cross}
    \pause
    \column{0.50\textwidth}
      \fignocaption{width = 0.65\textwidth}{figs/ex-strong-list-spec}
      \vspace{-0.60cm}
      \fignocaption{width = 0.20\textwidth}{figs/green-check}
      \vspace{-1.20cm}
      \uncover<3->{
	\begin{center}
	  \red{\footnotesize ``Strong List Specification''不允许}
	\end{center}
      }
  \end{columns}
\end{frame}
%%%%%%%%%%%%%%%%%%%%

%%%%%%%%%%%%%%%%%%%%
\begin{frame}{}
  \centerline{\Huge \teal{Jupiter}}
\end{frame}
%%%%%%%%%%%%%%%%%%%%

%%%%%%%%%%%%%%%%%%%%
\begin{frame}{}
  \centerline{\large $\blue{(n+1)}\; \text{replicas} \triangleq \blue{(n)}\; \text{\red{Client}} + \blue{(1)}\; \text{\red{Server}}$~\ncite{Nichols:UIST95}}

  \begin{center}
    \resizebox{0.50\textwidth}{!}{\input{tikz/jupiter-cs-tikz}}
  \end{center}

  \vspace{-0.60cm}
  \uncover<2->{
    \begin{center}
      \red{Server 负责将所有操作序列化} \\[6pt]
      用户操作 $\xrightarrow[\text{\hl{立即返回}}]{\quad \text{提交} \quad}$ Client 
      $\xrightarrow[\text{更新操作}]{\quad \text{FIFO} \quad}$ Server $\xrightarrow[\text{更新操作}]{\quad \text{FIFO} \quad}$ 其它Clients 
    \end{center}
  }
\end{frame}
%%%%%%%%%%%%%%%%%%%%

%%%%%%%%%%%%%%%%%%%%
\begin{frame}{}
  \begin{center}
    {\large \red{挑战:} \hl{如何处理并发操作带来的冲突?}} \\[3pt] \pause
    {\small \teal{(The server is not drawn.)}}
  \end{center}

  \vspace{-0.80cm}
  \begin{columns}
    \column{0.40\textwidth}
      \begin{center}
	\input{tikz/no-ot-tcs06-tikz}
      \end{center}
    \column{0.40\textwidth}
      \begin{center}
	\input{tikz/ot-tcs06-tikz}
      \end{center}
  \end{columns}

  \vspace{-0.30cm}
  \begin{center}
    \uncover<5->{\large \red{解决方案:} \hl{操作转换} (Operational Transformation; OT)\\\ncite{Ellis:SIGMOD89}}
  \end{center}
\end{frame}
%%%%%%%%%%%%%%%%%%%%

%%%%%%%%%%%%%%%%%%%%
\begin{frame}{}
  \fignocaption{width = 0.40\textwidth}{figs/ot}

  \begin{equation*}
    \text{\hl{交换律:}}\;\; \resizebox{0.35\textwidth}{!}{$\sigma; o_1; o_2' \equiv \sigma; o_2; o_1'$}
  \end{equation*}

  \centerline{\ncite{Ellis:SIGMOD89}}
\end{frame}
%%%%%%%%%%%%%%%%%%%%

%%%%%%%%%%%%%%%%%%%%
\begin{frame}{}
  \begin{center}
    \red{\large 问题: 如果副本节点之间偏离 \teal{\small (diverge)} 多个 ($\ge 2$) 操作, 怎么办?} \\[8pt]

    \resizebox{0.75\textwidth}{!}{\input{tikz/ot-diverge-two}}
  \end{center}
\end{frame}
%%%%%%%%%%%%%%%%%%%%

%%%%%%%%%%%%%%%%%%%%
\begin{frame}{}
  \begin{center}
    \hl{核心问题:} \\[8pt]
    当副本节点$r$接收到由Server转发的操作$o$时,\\
    如何对$o$执行操作转换?
  
    \pause
    \vspace{0.80cm}
    \red{\large 与哪些操作进行转换?以什么顺序进行转换?}

    \pause
    \vspace{0.80cm}

    \hl{核心思想:} \\[3pt]
    \begin{columns}
      \column{0.72\textwidth}
	\begin{enumerate}
	  \item 与``$r$ 上已执行且与 $o$ \purple{并发}的操作''进行转换
	  \item 以``由 Server 确定的\purple{序列化}顺序''进行转换 
	\end{enumerate}
    \end{columns}
  \end{center}
\end{frame}
%%%%%%%%%%%%%%%%%%%%

%%%%%%%%%%%%%%%%%%%%
\begin{frame}{}
  \begin{center}
    {\large 利用数据结构 \hl{$2D$ 状态空间}~\ncite{Xu:CSCW14} \\
    控制何时以及如何执行``操作转换''}
  \end{center}

  \fignocaption{width = 0.60\textwidth}{figs/ot-diverge-two}

  \begin{center}
    $2D$: {\textsc{Local}} \emph{vs.} \textsc{Global}
  \end{center}
\end{frame}
%%%%%%%%%%%%%%%%%%%%

%%%%%%%%%%%%%%%%%%%%
\begin{frame}{}
  \centerline{\large 每个 \red{Client} 维护一个 $2D$ 状态空间}

  \fignocaption{width = 0.70\textwidth}{figs/jupiter-illustration}

  \centerline{\large \red{Server} 维护 $n$ 个 $2D$ 状态空间, 与 $n$ 个 Clients 对应}
\end{frame}
%%%%%%%%%%%%%%%%%%%%

%%%%%%%%%%%%%%%%%%%%
\begin{frame}{}
  \begin{center}
    \hl{\Huge Mismatch!}

    \vspace{1.00cm}
    {\large \blue{\wlspec{}} 所规定的\red{全局性质}}

    \vspace{0.20cm}
    \fignocaption{width = 0.40\textwidth}{figs/mismatch}
    \vspace{0.20cm}

    {\large \blue{Jupiter} 协议中, 每个副本节点所维护的\red{局部视图}}
  \end{center}
\end{frame}
%%%%%%%%%%%%%%%%%%%%

%%%%%%%%%%%%%%%%%%%%
\begin{frame}{}
  \centerline{\LARGE \teal{CJupiter (Compact Jupiter)}}

  \pause
  \vspace{1.0cm}
  \begin{ctheorem}[等价性]
    在相同的操作调度 \emph{\teal{\small (schedule)}} 下, 
    \emph{CJupiter} 与 \emph{Jupiter} 中的对应副本节点的行为 \emph{\teal{\small (behavior; {\small 状态序列})}} 是相同的。
  \end{ctheorem}
\end{frame}
%%%%%%%%%%%%%%%%%%%%

%%%%%%%%%%%%%%%%%%%%
\begin{frame}{}
  \begin{center}
    {\large CJupiter 为每个副本节点维护一个 \hl{\red{$n$-ary} \blue{有序}状态空间}}
  \end{center}

  \begin{center}
    \resizebox{0.40\textwidth}{!}{\input{tikz/cjupiter-allinone}}
  \end{center}

  \pause
  \begin{center} 
    每个node可以有 \red{$>2$} 条出边 \\[5pt] 
    每个node的所有出边按照其上标记操作的``序列化''顺序排成\red{全序}
  \end{center}
\end{frame}
%%%%%%%%%%%%%%%%%%%%

%%%%%%%%%%%%%%%%%%%%
\begin{frame}{}
  \begin{center}
    \begin{prop}[Compactness of CJupiter]
      {\large \emph{CJupiter} 所维护的 $(n+1)$ 个$n$-ary有序状态空间是相同的。}
    \end{prop}

    \resizebox{0.45\textwidth}{!}{\input{tikz/cjupiter-allinone-path}}

    \pause
    每个副本节点的行为对应于该状态空间中的一条\hl{\red{路径}} \\[3pt]
    \pause
    \red{单个$n$-ary有序状态空间包含了系统所有可能的状态}
  \end{center}
\end{frame}
%%%%%%%%%%%%%%%%%%%%

%%%%%%%%%%%%%%%%%%%%
\begin{frame}{}
  \centerline{\teal{\LARGE CJupiter 满足 Weak List Specification}}
\end{frame}
%%%%%%%%%%%%%%%%%%%%

%%%%%%%%%%%%%%%%%%%%
\begin{frame}{}
  \begin{center}
    {\large 关注某个$n$-ary有序状态空间, \hl{三步骤}证明\red{``状态对兼容性''}}
  \end{center}

  \fignocaption{width = 0.45\textwidth}{figs/cjupiter-allinone-path}

  \pause
  \begin{center}
    \red{反证法、数学归纳法、分情形分析法}
  \end{center}
\end{frame}
%%%%%%%%%%%%%%%%%%%%

%%%%%%%%%%%%%%%%%%%%
\begin{frame}{}
  \centerline{\circled{1} 任取两个状态节点 $v_1$和$v_2$}

  \begin{clemma}[LCA (Lowest Common Ancestor)]
    $n$-ary 有序状态空间中的任意一对状态节点都有\red{唯一的}最近公共祖先。
  \end{clemma}

  \begin{columns}
    \column{0.60\textwidth}
      \fignocaption{width = 0.50\textwidth}{figs/lca}
      \column{0.30\textwidth}
	\[
	  v_0 = \text{LCA}(v_1, v_2)
	\]
  \end{columns}
\end{frame}
%%%%%%%%%%%%%%%%%%%%

%%%%%%%%%%%%%%%%%%%%
\begin{frame}{}
  \centerline{\circled{2} 考虑从 $v_0 = \text{LCA}(v_1, v_2)$ 到 $v_1$ 和 $v_2$ 的两条路径}

  \begin{clemma}[Disjoint Paths]
    路径 $P_{v_0 \leadsto v_1}$ 上包含的操作集 $O_{v_0 \leadsto v_1}$ 与路径 $P_{v_0 \leadsto v_2}$ 上包含的操作集
    $O_{v_0 \leadsto v_2}$ 不相交。
  \end{clemma}

  \begin{columns}
    \column{0.60\textwidth}
	\fignocaption{width = 0.50\textwidth}{figs/disjoint-paths}
      \column{0.30\textwidth}
	\[
	  v_0 = \text{LCA}(v_1, v_2)
	\]
  \end{columns}
\end{frame}
%%%%%%%%%%%%%%%%%%%%

%%%%%%%%%%%%%%%%%%%%
\begin{frame}{}
  \centerline{\circled{3} 考虑两条路径上的状态}

  \begin{clemma}[Compatible Paths]
    $P_{v_0 \leadsto v_1}$ 上的任一状态 $v$ 与 $P_{v_0 \leadsto v_2}$ 上的任一状态 $v'$ 是兼容的。
  \end{clemma}

  \begin{columns}
    \column{0.60\textwidth}
	\fignocaption{width = 0.55\textwidth}{figs/compatible-paths}
      \column{0.40\textwidth}
	\[
	  v_0 = \text{LCA}(v_1, v_2)
	\]

	\pause
	\vspace{0.50cm}
	\begin{center}
	  \hl{$\therefore$ $v_1$ 和 $v_2$ 是兼容的}
	\end{center}
  \end{columns}
\end{frame}
%%%%%%%%%%%%%%%%%%%%
