% file: preamble.tex

\usepackage{xeCJK}
\usepackage{zhnumber}	% counters in Chinese
\usepackage{fontspec}
\usepackage{comment}
\usepackage{xifthen}
\usepackage{verbatim}

\usetheme{CambridgeUS} % try Madrid
\usecolortheme{beaver} % try beaver, dolphin, seahorse
% \usefonttheme[onlymath]{serif} % try "professionalfonts"
\usefonttheme{serif}  % standard font (same with that in ``standalone'')
% \setCJKmainfont{Microsoft YaHei} % try SimSun

\usepackage{amsmath, amsfonts, amssymb, mathtools, pifont}
\newcommand{\cmark}{\ding{51}}%
\newcommand{\xmark}{\ding{55}}%
\def\checkmark{\tikz\fill[scale=0.5](0,.35) -- (.25,0) -- (1,.7) -- (.25,.15) -- cycle;} 

\usepackage[normalem]{ulem} % strike through text
\newcommand{\soutthick}[1]{%
    \renewcommand{\ULthickness}{2.0pt}%
       \sout{#1}%
    \renewcommand{\ULthickness}{.4pt}% Resetting to ulem default
}

\setbeamersize{text margin left = 2em, text margin right = 1em}
\setbeamercolor{footnote mark}{fg = teal}
\setbeamertemplate{itemize items}[default]
\setbeamertemplate{enumerate items}[default]

% \hypersetup{colorlinks = false, urlcolor = DarkRed}

% \usepackage{perpage} %the perpage package
% \MakePerPage{footnote} %the perpage package command

\usepackage{tikz}
\usepackage{pgfplots}
\usetikzlibrary{arrows.meta, shapes, positioning, calc, chains, backgrounds, fit, mindmap, intersections}

\newcommand*{\circled}[2][]{\tikz[baseline=(C.base)]{
    \node[inner sep = 0pt] (C) {\vphantom{1g}#2};
    \node[draw, circle, inner sep = 3pt, yshift = 1pt, fill = red!40, opacity = 0.5]
        at (C.center) {\vphantom{1g}};}}

\theoremstyle{plain}
\newtheorem{cdef}{定义}[section]
\newtheorem{ctheorem}{定理}[section]
\newtheorem{clemma}{引理}[section]
\newtheorem{cquestion}{问题:}[section]
\newtheorem{cobservation}{观察}[section]
\newtheorem{prop}{命题}
\theoremstyle{proof}
\newtheorem{cproof}{证明}[section]

% for tables
\usepackage{multirow}
\newcommand{\innercell}[2]{\begin{tabular}{@{}#1@{}}#2\end{tabular}}
\usepackage{hhline}
%%%%%%%%%%%%%% for appendix %%%%%%%%%%%%%%%%
% http://www-ljk.imag.fr/membres/Jerome.Lelong/latex/appendixnumberbeamer.sty
% Reference: http://tex.stackexchange.com/questions/2541/beamer-frame-numbering-in-appendix
\usepackage{appendixnumberbeamer}
% Add total frame count to slides, optional. From Stefan,
% http://www.latex-community.org/forum/viewtopic.php?f=4&t=2173
\expandafter\def\expandafter\insertshorttitle\expandafter{%
  \insertshorttitle\hfill\insertframenumber\,/\,\inserttotalframenumber}
%%%%%%%%%%%%%% for appendix %%%%%%%%%%%%%%%%
\usepackage{graphicx, subcaption}

\usepackage{caption}
\DeclareCaptionLabelSeparator{none}{}
\captionsetup{labelsep = none}

\makeatletter
\let\@@magyar@captionfix\relax
\makeatother

\usepackage[export]{adjustbox}
% for fig without caption: #1: width/size; #2: fig file
\newcommand{\fignocaption}[2]{
  \begin{figure}[htp]
    \centering
    \includegraphics[#1]{#2}
  \end{figure}
}

% for fig without caption: #1: width/size; #2: fig file; #3: fig caption
\newcommand{\fig}[3]{
  \begin{figure}[htp]
    \centering
      \includegraphics[#1]{#2}
      \caption{#3}
  \end{figure}
}

% colors
\definecolor{DarkRed}{rgb}{0.55, 0.0, 0.0}
\newcommand{\red}[1]{\textcolor{red}{#1}}
\newcommand{\redoverlay}[2]{\textcolor<#2>{red}{#1}}
\newcommand{\green}[1]{\textcolor{green}{#1}}
\newcommand{\greenoverlay}[2]{\textcolor<#2>{green}{#1}}
\newcommand{\blue}[1]{\textcolor{blue}{#1}}
\newcommand{\blueoverlay}[2]{\textcolor<#2>{blue}{#1}}
\newcommand{\purple}[1]{\textcolor{purple}{#1}}
\newcommand{\cyan}[1]{\textcolor{cyan}{#1}}
\newcommand{\violet}[1]{\textcolor{violet}{#1}}
\newcommand{\lgray}[1]{\textcolor{lightgray}{#1}}
\newcommand{\teal}[1]{\textcolor{teal}{#1}}
\newcommand{\brown}[1]{\textcolor{brown}{#1}}

\newcommand{\wlspec}{$\mathcal{A}_{\text{weak}}$}

% colorded box
\newcommand{\rbox}[1]{\red{\boxed{#1}}}
\newcommand{\gbox}[1]{\green{\boxed{#1}}}
\newcommand{\bbox}[1]{\blue{\boxed{#1}}}
\newcommand{\pbox}[1]{\purple{\boxed{#1}}}

\usepackage[linewidth = 1pt, framemethod = TikZ]{mdframed}
\mdfsetup{frametitlealignment=\center}

\newcommand{\hl}[1]{\fcolorbox{yellow}{yellow!60}{#1}}

% for cite: #1: author; #2: conference #3: year
\newcommand{\citeinbeamer}[3]{{\tiny{\textcolor{blue}{[#1@#2'#3]}}}}

\usepackage[natbib = true, backend = bibtex, style = authoryear]{biblatex}
\setbeamertemplate{bibliography item}[article]
\renewcommand*{\bibfont}{\footnotesize}
\addbibresource{njucs-youth-salon-2018-report.bib}

\newcommand{\ncite}[1]{\violet{\footnotesize [\cite{#1}]}}

\newcommand{\term}[1]{\small (#1)}
\newcommand{\set}[1]{\{#1\}}
\newcommand{\question}[1]{\textcolor{red}{\centerline{#1}}}
\newcommand{\answer}[1]{\textcolor{blue}{\centerline {#1}}}
\newcommand{\alertred}[1]{\textcolor{red}{#1}}
\newcommand{\alertblue}[1]{\textcolor{blue}{#1}}
\newcommand{\todo}[1]{\textcolor{red}{\textbf{TODO:} #1}}
\newcommand{\mathbfblue}[1]{\textcolor{blue}{$\mathbf{#1}$}}

\newcommand{\hhref}[2]{\centerline{\href{#1}{\purple{\underline{#2}}}}}

\newcommand{\papertitle}{复制数据类型理论研究}

\newcommand{\thankyou}{
  \begin{frame}[noframenumbering]
    \begin{columns}
      \column{0.50\textwidth}
	\fignocaption{width = 0.80\textwidth}{figs/rdt-research-two-work}
      \column{0.50\textwidth}
	\fignocaption{width = 0.80\textwidth}{figs/thankyou}
    \end{columns}

    \begin{columns}[b]
      \column{0.30\textwidth}
	\fignocaption{width = 0.90\textwidth}{figs/hfwei-mail}
	\vspace{-0.20cm}
	\hhref{http://www.bigoh.net/wiki/index.php/user:Hengfeng-Wei}{hengxin@homepage}
      \column{0.30\textwidth}
	\fignocaption{width = 0.50\textwidth}{figs/github-icon}
	\vspace{-0.20cm}
	\hhref{https://github.com/hengxin}{hengxin@github}
      \column{0.30\textwidth}
	\fignocaption{width = 0.80\textwidth}{figs/hengxin-stack}
	\vspace{-0.20cm}
	\hhref{https://stackexchange.com/users/2055160/hengxin}{hengxin@stackexchange}
    \end{columns}
  \end{frame}
}
